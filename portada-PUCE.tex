%%%%%%%%%%%%%%%%%%%%%%%%%%%%%%%%%%%%
%% --- Portada
%%%%%%%%%%%%%%%%%%%%%%%%%%%%%%%%%%%%
\makeatletter
%%  Previos %%  Altura de la caja de título
\renewcommand\@ytitulo{0.67\paperheight}
%%  Previos %%  Logos de fondo
\renewcommand{\logofondo}[2][0.65\paperwidth]
    {\renewcommand\@logofondo{#2}
    \newcommand\@llogofondo{#1}}
%%  Previos %%  Longuitud de la caja de título
\renewcommand\@ltitulo{0.83\paperwidth}
%%%%%%%%%%%%%%%%%%%%%%%%%%%%%%%%%%%%
%%  Comando de portada
\renewcommand{\portada}{
    \pagenumbering{alph}
    \newgeometry{left=0cm,right=0cm,top=0cm,bottom=0cm}
    \thispagestyle{empty}
    \begingroup\clearpage\noindent
\begin{pspicture}(0,0)(0.99\paperwidth,\paperheight)%
%%  Imagen de fondo
    \ifthenelse{\equal{\@logofondo}{}}{}{
    \begin{psclip}{
        \psframe[linestyle=none](.27\paperwidth,.13\paperheight)
            (.98\paperwidth,.55\paperheight)}
        % \psframe(0.\paperheight,.0\paperheight)
        %     (1\paperheight,.99\paperheight)}
        \rput(0.64\paperwidth,.335\paperheight)
        {\includegraphics[width=\@llogofondo]{\@logofondo}}
    \end{psclip}}
%%  Título
    \rput(0.5\paperwidth,\@ytitulo)
    {\psshadowbox[framearc=0.25,linecolor=colorportada,
        shadowcolor=colorportada!50,framesep=0.5cm]
        {\begin{minipage}{\@ltitulo}\begin{spacing}{1.2}
        \ifthenelse{\equal{\@fasciculo}{}}
        {
        \ifthenelse{\equal{\@subtitulo}{}}
            {\Huge\centering \textbf{\textsc{\@titulo}}}
            {\Huge\centering \textbf{\textsc{\@titulo\\[2mm]
            \huge \@subtitulo}}}
        }
        {
        \ifthenelse{\equal{\@subtitulo}{}}
            {\Huge\centering \textbf{\textsc{\@titulo\\ \LARGE \@fasciculo}}}
            {\Huge\centering \textbf{\textsc{\@titulo\\ \huge \@subtitulo \\
                \LARGE \@fasciculo}}}
        }
        \end{spacing}\vspace{-\baselineskip}\end{minipage}
        }
    }
%%  Linea autor
    \psframe*[linecolor=colorportada](-.1,-.1)
        (1.01\paperwidth,.12\paperheight)
        \uput{5mm}[l](.99\paperwidth,.08\paperheight)
        {\huge\color{white} \textbf{\@autor}}
%%  Recuadro superior
    \psframe*[linecolor=colorportada]
        (-.1,.76\paperheight)(1.01\paperwidth,1.01\paperheight)
%%  Parte superior
    \rput(0.4\paperwidth,.88\paperheight)
        {\includegraphics[width=\@lplogouno]{\@logouno}}
    \psline[linecolor=white,linewidth=2pt]
        (.67\paperwidth,.96\paperheight)
        (.67\paperwidth,.80\paperheight)
    \uput[r](.67\paperwidth,.92\paperheight){
        \begin{minipage}{0.12\paperheight}
            \huge\color{white} \textbf{\@series}
        \end{minipage}}
    \uput[r](.67\paperwidth,.84\paperheight){
        \begin{minipage}{0.12\paperheight}
            \color{white}
            \scalebox{6}{\textbf{\@numero}}
        \end{minipage}}
%%  Nombre escuela
    \uput[u](.13\paperwidth,.13\paperheight){
        \rotatebox{90}{\begin{minipage}{0.44\paperheight}
            \huge \textbf{\textsc{Escuela de Ciencias Físicas y Matemática}}
        \end{minipage}}}
\end{pspicture}
\endgroup
\restoregeometry
\ifthenelse{\equal{\@npblanco}{true}}{}
    {\cleartooddpage[\thispagestyle{empty}\vspace*{\fill}
    \textcolor{gray!50}{Esta página ha sido dejada intencionalmente en blanco.}
    \par\vspace*{\fill}]}
}
%%%%%%%%%%%%%%%%%%%%%%%%%%%%%%%%%%%%
%% --- Portadilla
%%%%%%%%%%%%%%%%%%%%%%%%%%%%%%%%%%%%
%%  Este comando genera la portadilla y la hoja de créditos.
\renewcommand{\portadilla}{
    \newgeometry{left=3cm,right=3cm,top=3cm,bottom=3cm}
    \begin{spacing}{1.0}
    \thispagestyle{empty}
    \setcounter{page}{1}
    \pagenumbering{roman}
\begingroup
    \centering
%%  Serie
    \begin{spacing}{1.3}
        \Large\scshape \@series
    \end{spacing}
%%  Autor
    \vspace{0.21\textheight}
        {\large\scshape \@autorcorto}\\[1\baselineskip]
%%  Titulo entre lineas
        \begingroup
        \color{colortext}
        \rule{\linewidth}{1.6pt}\\[-\baselineskip]\vspace{3pt}
        \rule{\linewidth}{0.4pt}\\[0.5\baselineskip]
        \begin{spacing}{1}
        \ifthenelse{\equal{\@fasciculo}{}}
        {
        \ifthenelse{\equal{\@subtitulo}{}}
            {\LARGE \textsc{\@titulo}}
            {{\LARGE \textsc{\@titulo}}\\[\baselineskip]
            {\Large\scshape \@subtitulo}}
        }
        {
        \ifthenelse{\equal{\@subtitulo}{}}
            {\LARGE \textsc{\@titulo}\\[\baselineskip]{\large\@fasciculo}}
            {{\LARGE \textsc{\@titulo}}\\[\baselineskip]
            {\scshape \Large\@subtitulo}\\[\baselineskip]{\large\@fasciculo}}
        }
        \end{spacing}
        \vspace*{-0.5\baselineskip}
        \rule{\linewidth}{0.4pt}\\[-\baselineskip]\vspace{4.5pt}
        \rule{\linewidth}{1.6pt}
        \endgroup
%%  Logos
    \par\vfill
    \includegraphics[width=\@lclogodos]{\@logodos}
    \par\vspace*{0.001\textheight}
%%  Página de créditos
\newpage
\thispagestyle{empty}
\raggedright
%%  Serie, título y autor
    \vspace*{\baselineskip}
    {\textbf{\@serie\ No. \@numero}}\\[0.6\baselineskip]
    {\scshape
    \ifthenelse{\equal{\@fasciculo}{}}
    {
    \ifthenelse{\equal{\@subtitulo}{}}
        {\@titulo}
        {\@titulo\@sepsubtitulo\ \@subtitulo}
        \\[0.6\baselineskip]
    }
    {
    \ifthenelse{\equal{\@subtitulo}{}}
        {\@titulo\\[0.2\baselineskip] \@fasciculo}
        {\@titulo\@sepsubtitulo\ \@subtitulo\\[0.2\baselineskip]\@fasciculo}
        \\[0.6\baselineskip]
    }
    }
    {\@autor}\par
    \small\vspace*{4\baselineskip}
%%  Créditos
    \ifthenelse{\equal{\@asistente}{}}{}
        {\textbf{Asistentes}: \@asistente \\[1mm]}
    \ifthenelse{\equal{\@editor}{}}{}
        {\textbf{Responsable de la Edici\'on}: \@editor \\[1mm]}
    \ifthenelse{\equal{\@asisedicion}{}}{}
        {\textbf{Asistente de Edici\'on}: \@asisedicion \\[1mm]}
    \ifthenelse{\equal{\@revision}{}}{}
        {\textbf{Revisi\'on Acad\'emica}: \@revision \\[1mm]}
    \vspace*{4\baselineskip}
%%  Registro e ISBN
    Registro de derecho autoral No. \@regautoral\\[1mm]
    ISBN: \@ISBN\par
    \vspace*{4\baselineskip}
%%  Publicado por
    \ifthenelse{\equal{\@publicado}{}}{}
    {Publicado \@publicado\par
    \vspace*{4\baselineskip}}
%%  Edición e impresión
    \@edicion\ edici\'on: \@fechaedicion\\[1mm]
    \ifthenelse{\equal{\@impresion}{}}{}
    {
    \ifthenelse{\equal{\@correcciones}{}}
        {\@impresion\ impresi\'on: \@fechaimpresion}
        {\@impresion\ impresi\'on (con correcciones): \@fechaimpresion}\par
    }
    \vspace*{4\baselineskip}
%%  Derechos
\copyright\ \@derechos\ \@fechapub
%%  Nota
\ifthenelse{\equal{\@nota}{}}{}
    {\par\vspace*{4\baselineskip}\@nota}
\endgroup
    \end{spacing}
    \restoregeometry
    \cleartooddpage[\thispagestyle{empty}]
}

\makeatother




\iffalse


%%%%%%%%%%%%%%%%%%%%%%%%%%%%%%%%%%%%
%% --- Contraportada
%%%%%%%%%%%%%%%%%%%%%%%%%%%%%%%%%%%%
%%  Altura de la caja de título
\newcommand\@ytexto{0.55\paperheight}
\newcommand{\ytexto}[1]{\renewcommand{\@ytexto}{#1}}
%%  Medidas
\newlength{\xqr}
\newlength{\xisbn}
\setlength{\xqr}{0.33\paperwidth-0.45in}
\setlength{\xisbn}{0.66\paperwidth-0.75in}
%%  Este comando genera la contraportada.
\newcommand{\contraportada}[3]{
\ifthenelse{\equal{\@npblanco}{true}}{}
    {\cleartooddpage[\thispagestyle{empty}\vspace*{\fill}
    \textcolor{gray!50}{Esta página ha sido dejada intencionalmente en blanco.}
    \par\vspace*{\fill}]}
    \newgeometry{left=0cm,right=0cm,top=0cm,bottom=0cm}
    \thispagestyle{empty}
\begingroup
    \clearpage
    \noindent
    \begin{pspicture}(0,0)(\paperwidth,\paperheight)
\rput(0.5\paperwidth,\@ytexto)
    {
        \begin{minipage}{.75\paperwidth}\begin{spacing}{1.2}
        \slshape #3
        \end{spacing}\vspace{-\baselineskip}\end{minipage}
    }
%%  Código QR e ISBN
\rput(\xqr,.09\paperheight){\psbarcode{#2}
        {width=1 height=1}{qrcode}}
\rput(\xisbn,.09\paperheight){\psbarcode{\@ISBN}
        {includetext guardwhitespace}{isbn}}
%%  Linea 1
    \psframe*[linecolor=colorp](-.1,.245\paperheight)
        (1.01\paperwidth,.295\paperheight)
\rput(.5\paperwidth,.27\paperheight){\LARGE\color{white} \textbf{#1}}
%%  Linea título
\psframe*[linecolor=colorp](-.1,.72\paperheight)
        (1.01\paperwidth,.77\paperheight)
\rput(.5\paperwidth,.745\paperheight){\huge\color{white} \textbf{\@titulo}}
    \end{pspicture}
\endgroup
\restoregeometry
}

\fi